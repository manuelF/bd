%BEGIN COPYPASTE EL INFORME DEL INFO
\documentclass[10pt, a4paper,english,spanish]{article}
\usepackage{subfig}

\parindent=20pt
\parskip=8pt
\usepackage[width=15.5cm, left=3cm, top=2.5cm, height= 24.5cm]{geometry}

\usepackage{ccfonts,eulervm} 
\usepackage[T1]{fontenc}
\usepackage{epigraph}
\usepackage{amsmath}
\usepackage{amsfonts}
\usepackage{amssymb}
\usepackage{fancyhdr}
\usepackage[activeacute, spanish]{babel}
\usepackage{cancel}
\usepackage[utf8]{inputenc}
\usepackage{algorithm}
\usepackage{algpseudocode}
\usepackage{afterpage}
\usepackage{caratula}
\usepackage{url}
\usepackage{fancyhdr}
\usepackage{listings}
\usepackage{ulem}
\usepackage{dashrule}
\usepackage{pdflscape}

\floatname{algorithm}{Algoritmo}

\newtheorem{theorem}{Teorema}[section]
\newtheorem{lemma}[theorem]{Lema}
\newtheorem{proposition}[theorem]{Proposici\'on}
\newtheorem{corollary}[theorem]{Corolario}

\newcommand{\Var}{\textbf{var }}
\newcommand{\True}{\textbf{true }}
\newcommand{\False}{\textbf{false }}
\newcommand{\Break}{\textbf{break }}
\newcommand{\Continue}{\textbf{continue }}
\newcommand{\Param}{\textbf{param }}

\newenvironment{proof}[1][Demostraci\'on]{\begin{trivlist}
\item[\hskip \labelsep {\bfseries #1}]}{\end{trivlist}}
\newenvironment{definition}[1][Definici\'on]{\begin{trivlist}
\item[\hskip \labelsep {\bfseries #1}]}{\end{trivlist}}
\newenvironment{example}[1][Ejemplo]{\begin{trivlist}
\item[\hskip \labelsep {\bfseries #1}]}{\end{trivlist}}
\newenvironment{remark}[1][Observaci\'on]{\begin{trivlist}
\item[\hskip \labelsep {\bfseries #1}]}{\end{trivlist}}

\newcommand{\qed}{\nobreak \ifvmode \relax \else
      \ifdim\lastskip<1.5em \hskip-\lastskip
      \hskip1.5em plus0em minus0.5em \fi \nobreak
      \vrule height0.75em width0.5em depth0.25em\fi}

\parindent 0em
\algrenewcommand{\algorithmiccomment}[1]{//\textit{#1} }

\lstset{language=sql,numbers=left,tabsize=2,
	morekeywords={BEGIN,DECLARE,FOR,CREATE,PROCEDURE,RAISEERROR,EACH,ROW,BEFORE,AFTER,MINUS,IF},
	breaklines=true,breakatwhitespace=true}

\pagestyle{fancy}
\thispagestyle{fancy}
\addtolength{\headheight}{1pt}
\lhead{BD - TP1}
\rhead{Grupo 1}
\cfoot{\thepage}
\renewcommand{\footrulewidth}{0.4pt}
\newcommand{\hblacksquare}{\hfill \blacksquare}
%FIN COPYPASTE EL INFORME DEL INFO
\begin{document}

\materia{Bases de datos}
\submateria{Primer Cuatrim\'estre de 2013}
\titulo{Trabajo Pr\'actico 2}
\subtitulo{Informe}
\grupo{Grupo 1}
\integrante{Pablo Gauna}{334/09}{gaunapablo@gmail.com}
\integrante{Juli\'an Sackmann}{540/09}{jsackmann@gmail.com}
\integrante{Manuel Ferreria}{199/10}{m.ferreria@gmail.com}
\integrante{Juan Pablo Darago}{272/10}{jpdarago@gmail.com}

\maketitle
\pagebreak

\tableofcontents
\pagebreak

\section{Investigación sobre Oracle}

Desde su versión 8i, el motor de base de datos \texttt{Oracle} utiliza tres buffer pools para tener bloques de datos. Un resumen sucinto de la arquitectura de buffers en \texttt{Oracle} puede verse en la figura \ref{fotito}.

\begin{figure}[H]\centering
	\includegraphics[scale=0.5]{keepRecycle.jpg}
	\caption{Los tres buffers de \texttt{Oracle 8i}}
	\label{fotito}
\end{figure}
\subsection{Uso}
Para utilizar los distintos buffer pools, se marcan objetos de la base de datos de tal forma que quedan asociados a un pool particular. Luego de esto, todos los bloques asociados a ese objeto (un bloque no puede tener elementos de distintos objetos) irá siempre a ese buffer\footnote{http://docs.oracle.com/cd/B10500\_01/server.920/a96533/memory.htm}. 

Es digno observar que estos buffers no tienen un comportamiento automático. Es el usuario quien debe encargarse de marcar qué páginas irán a qué buffer para que los buffers acá definidos sean utilizados. 

\subsubsection{Ejemplo}
\begin{Verbatim}[xleftmargin=-3em]
	CREATE TABLE t1 (
		my_date   DATE NOT NULL,
		my_number NUMBER(12,10) NOT NULL,
		my_row    NUMBER(12) NOT NULL)
	STORAGE (BUFFER_POOL KEEP);	
\end{Verbatim}

\subsection{Default pool}
Es el espacio de memoria que se utiliza por defecto para los bloques que pertenecen a objetos que no fueron marcados específicamente para ser usados con otras tablas.

\newpage

\subsection{Keep pool}
El buffer \texttt{keep} se usa para almacenar bloques pertenecientes a tablas e índices que son utilizados muy frecuentemente. 

Este buffer pool es particularmente eficiente a la hora de ``proteger'' a los bloques de memoria más frecuentemente utilizados de los nocivos efectos de desalojo masivo que tienen las implementaciones más triviales de \textit{full table scan}. Generalmente, al realizar una de estas consultas se requiere mucha memoria, lo que termina forzando que un gran número de bloques de tablas frecuentemente utilizadas sean descartadas y sea necesario volver a traerlos de disco si se quieren volver a utilizar. Es por esto que \texttt{Oracle} permite marcar tablas para que estas vayan al buffer \texttt{keep}. De esta forma se intenta evitar su pérdida cuando deben ser desalojados por alguna query.

Notar que este método no es 100\% efectivo, dado que muy probablemente la suma de los bloques de todas las tablas marcadas como \texttt{keep} (así sea una sola) supere el tamaño del buffer. Con lo cual en algún momento es posible que sea necesario guardar un bloque en el buffer \texttt{keep} cuando este esté lleno, con lo que va a tener que descartarse algún bloque mediante alguna política de reemplazo (en \texttt{Oracle}, generalmente \textit{LRU}). 

En general se recomienda que el buffer \texttt{keep} sea utilizado para objetos pequeños\footnote{http://www.exploreoracle.com/2009/04/02/keep-buffer-pool-and-recycle-buffer-pool/}.

\subsection{Recycle pool}
A diferencia del \texttt{keep} buffer, el \texttt{recycle} buffer debe ser utilizado con bloques pertenecientes a tablas que son raramente accedidos. Los bloques que están en este pool son descartados rápidamente una vez que dejan de ser utilizados\footnotemark[\value{footnote}]. Esto es porque en Oracle existe una opción \textit{ASM} que, usada en este caso, permite realocar la memoria del \texttt{recycle} a otro componente del SGA (\textit{System Global Area}, un componente que forma parte del sistema de administración de memoria del DBMS).

Por recomendación de \texttt{Oracle}\footnote{http://docs.oracle.com/cd/B28359\_01/server.111/b28274/memory.htm\#autoId29}, el tamaño del \texttt{recycle} pool debe ser lo suficientemente chico como para no ocupar RAM que podría ser mejor destinada a otro pool (como lo bloques se desalojan relativamente rápido, es mucho mejor usado el espacio en el default o \texttt{keep} pool), pero lo suficientemente grande como para que un bloque no sea desalojado en medio de una transacción (supongamos una transacción que levanta un bloque de un objeto marcado de \texttt{recycle}, luego lo procesa junto con otros y luego lo escribe. Si el \texttt{recycle} buffer es demasiado pequeño y fue necesario desalojar el bloque durante el procesado, a la hora de grabarlo va a ser necesario volver a levantarlo, lo que es mucho más ineficiente).

Históricamente (en \texttt{Oracle 8}), el \texttt{recycle} pool comenzó como un buffer donde alojar bloques \textit{transitorios}\footnote{http://www.praetoriate.com/t\_\%20tuning\_recycle\_pool.htm}. Un bloque \textit{transitorio} es aquel que se lee como parte de un \textit{full table scan} y es poco probable que vuelva a ser utilizado en el futuro cercano. Este concepto luego fue extendido de bloques transitorios a cualquier bloque perteneciente a objetos particulares que el dba sepa que serán utilizados muy esparsamente.

\newpage
%\input{bufferPool.tex}
\newpage

\section{Tests de performance}
Para correr los tests de performance se utiliz\'o el paquete \texttt{ubadb.external.bufferManagement} provisto en el
motor para generar trazas v\'alidas que utilizamos como inputs, mediante las clases \texttt{MainEvaluator} y \texttt{MainTraceGenerator}.

Para generar las trazas, utilizamos m\'etodos similares a los implementados en \texttt{MainTraceGenerator}. Esto es
utilizar distintos objetos para generar trazas que simulen accesos normales a una base de datos, siguiente un cierto
patr\'on. Es decir, podemos generar por ejemplo, Filescans de \textit{n} accesos, joins de tipo \textit{BNLJ}, con
tama\~nos parametrizados o accesos a indices del estilo \textit{B+ Tree Clustered} o \textit{Hash}.

Una vez generadas estas trazas corrimos el evaluador (tambi\'en incluido en el c\'odigo) y observamos el
desempe\~no de nuestras implementaciones de single y m\'ultiple buffer pools. Para realizar mediciones m\'as
precisas, comparamos los \textit{hitrates} de acceso a disco considerando casos interesantes de las trazas generadas.
Con esta informaci\'on de los \textit{hitrate}, pudimos generar tablas comparativas que se pueden apreciar en la secci\'on \ref{secTablas}.


\subsection{Resultados de la experimentaci\'on}\label{secTablas}

Fue de interes buscar casos patol\'ogicos para comprobar donde una pol\'itica de
buffers presentara ventajas sobre la otra (buffer compartido vs buffer m\'ultiples).
Nuestra herramienta para comprobar accesos m\'as eficientes es el \textit{hitrate},
es decir, cuantas veces se puede acceder a una p\'agina en memoria por cada vez que
la tenemos que acceder a disco. Decimos que el \textit{hitrate} es ideal cuando
tiende a 100\%, el m\'aximo te\'orico. Entre m\'as alto, mejor.

Una simulaci\'on que hicimos fue una union \texttt{BNLJ} con buffers muy peque\~nos.
Tenemos una tabla de dos p\'aginas, que la marcamos como \texttt{KEEP} y tenemos
una tabla de mil p\'aginas, que la marcamos como \texttt{RECYCLE}.

En total, designamos dos buffer pools con capacidad de dos p\'aginas cada uno
y comparamos contra varios tama\~nos de single buffer pools distintos.

\begin{table}[H]\centering
    \begin{tabular}{l || c}
    \multicolumn{1}{c ||}{\large{\textbf{Estrategia}}} & \large{\textbf{Hitrate}} [\%] \\
    \hline
                Multiple Buffer Pools KEEP=2 RECYCLE=1 & 49.95       \\
                Single Buffer Pool SIZE=2              & 17.55       \\
                Single Buffer Pool SIZE=4              & 37.70       \\
                Single Buffer Pool SIZE=10             & 48.15       \\
                Single Buffer Pool SIZE=100            & 49.50       \\
                Single Buffer Pool SIZE=200            & 49.70       \\
    \end{tabular}
\end{table}

\begin{figure}[H]\centering
    \includegraphics[scale=0.4]{BNLJ.png}
    \caption{Gráfico de \textit{hitrate} con respecto a buffer utilizado}
    \label{grafiquito}
\end{figure}

%MANU, SI QUERES HACER GRAFICOS TAN TOP COMO LOS NUESTROS, CLAVATE UN http://www.chartgo.com/. =D =D =D.
%BESITOSSSSSSS Juampi y Juli.

Como se puede observar, con sólamente 3 p\'aginas y una administraci\'on inteligente, la eficiencia
de este mecanismo de m\'ultiples buffers supera a un buffer pool \'unico con tama\~no m\'as de 50 veces mayor.

\ \\

Otro experimento realizado consistió en simular los accesos a una base de datos realizados
por distintos usuarios. El caso de uso que modelamos ac\'a es realizar accesos a dos tablas,
una chica de 100 p\'aginas y una mucho mas grande, de 10000 p\'aginas. Esto representa de manera
razonablemente fidedigna un \textit{Facebook} simplificado. Se cuenta con una tabla (relativamente) chica de usuarios y 
una tabla gigantesca de interacciones, como pueden ser los \textit{Like} o las publicaciones. 
Se van a acceder de manera muy frecuente a las tablas con la informaci\'on de los usuarios, pero
ocasionalmente a las de interacciones, y de forma esparsa (no secuencial).

Como la tabla de usuarios va a ser accedida frecuentemente, la pondremos en KEEP, mientras
que las de interacciones las vamos a poner en RECYCLE.

Generamos la traza haciendo 200 accesos aleatorios a las 100 p\'aginas de usuario, y de forma
intercalada, generamos 1000 accesos aleatorios a las 10000 p\'aginas de interacciones.

Los resultados obtenidos son:

\begin{table}[H]\centering
    \begin{tabular}{l || c}
    \multicolumn{1}{c ||}{\large{\textbf{Estrategia}}}  & \large{\textbf{Hitrate}} [\%] \\
    \hline
                Multiple Buffer Pools KEEP=20 RECYCLE=1 & 3.08   \\
                Multiple Buffer Pools KEEP=30 RECYCLE=1 & 4.09   \\
                Single Buffer Pools SIZE=21             & 0.66   \\
                Single Buffer Pools SIZE=31             & 0.66   \\
                Single Buffer Pools SIZE=51             & 1.08   \\
                Single Buffer Pools SIZE=101            & 3.01   \\
                \end{tabular}
            \end{table}

\begin{figure}[H]\centering
    \includegraphics[scale=0.4]{RandomAccess.png}
    \caption{Gráfico de \textit{hitrate} para accesos aleatorios con respecto a buffer utilizado}
    \label{grafiquito2}
\end{figure}

Nuevamente, podemos observar que es necesario grandes tama\~nos del single buffer para
poder capturar algo de performance. Hay que tener en cuenta que son accesos aleatorios,
sin patr\'on definido y con que los buffers en memoria puedan capturar un peque\~no 
porcentaje de los accesos m\'as frecuentes basta para traer considerables mejoras al sistema.

\ \\

También testeamos la eficiencia de un INLJ. Implementamos un generador de traces que simulara un INLJ entre dos tablas.
Este generador toma por cada página de la primera tabla y se le indica cuantas tuplas está guardando en cada página. 
Por cada una de estas tuplas hace una busque en un índice (el cual se le pasa como  parámetro la cantidad de paginas
que tiene y la altura), haciendo k lecturas random de alguna página, donde k es la altura del árbol. También simula
la búsqueda de páginas en la segunda tabla, se le pasa al generador un número mínimo y máximo de páginas que se 
podrían necesitar buscar en esta segunda página por cada tupla, y se hace R lecturas a de páginas random de la
segunda tabla (donde R es un random entre [min,max] especificados como parámetros.



Generamos la traza haciendo un scanFile de una primer tabla de 100 páginas de tama\~{n}o, por cada una de las paginas
supusimos que tenía 5 tuplas adentro y que cada una de esas tuplas podría encontrar entre 0 y 5 páginas
para leer de la tabla indexada que cuenta con 1000 paginas (con un índice de 5 páginas y de altura 2)

\begin{table}[H]
        \begin{tabular}{l||l}
    \large{\textbf{Estrategia}}                             & \large{\textbf{Hitrate}} [\%] \\
    \hline
                Multiple Buffer Pools KEEP=5 RECYCLE=2		&	47.52	\\
                Multiple Buffer Pools KEEP=50 RECYCLE=2		&	47.52	\\
                Multiple Buffer Pools KEEP=50 RECYCLE=50 	&	49.80	\\
                Single Buffer Pools SIZE=21             	&	31.12	\\
                Single Buffer Pools SIZE=31             	&	40.94	\\
                Single Buffer Pools SIZE=51             	&	44.66	\\
                Single Buffer Pools SIZE=101            	&	45.80	\\
                \end{tabular}
            \end{table}
\begin{figure}[H]\centering
    \includegraphics[scale=0.4]{INLJ.png}
    \caption{Gráfico de \textit{hitrate} para INLJ con respecto a buffer utilizado}
    \label{grafiquito3}
\end{figure}

Se puede observar que múltiple buffer pool llega rápidamente a su máximo hitrate una vez que keep alcanzada
la cantidad de páginas que cuenta el índice, debido a que tiene los índices en el buffer keep 
y a las dos tablas en recycle. Esto se debe a que para la tabla peque\~{n}a, donde se hace el scanFile, 
es evidente que no va a dar ni un solo hit por lo que es un claro candidato a recycle, y la tabla con índice,
 debido a sus tama\~{n}o, es muy poco probable que vuelva a ser llamada la misma página para que de un hit
 a pesar de tener un tama\~{n}o conciderable.\\

A diferencia del MBP, el SBP comparte las páginas del índice con las otras tablas, por lo que el hitrate
bajo considerablemente, esto puede solucionarse fácilmente manteniendo las páginas del índice en estado
$"REQUEST"$ y liberarlas todas juntas al final, en este caso la performance de SBP sacaría equivalente a la de MBP.

\ \\

Se puede decir que estos experimentos son conociendo el algoritmo del Or\'aculo,
pero es claramente posible comprobarlo en motores reales. Si se contasen
con estadísiticas de uso de las tablas, sería posible asignarles de manera
din\'amica a que pool van a pertenecer cuando se carguen, e incluso es
posible preveer la performance de las operaciones de acuerdo a que pool
se le asigne, considerando como se la va a emplear. Es decir, no estaría
igual de bien asignar una tabla a \texttt{KEEP} si se la va a usar para \textit{filescan}
una sola vez, que asignar a \texttt{KEEP} una tabla que va a ser usada cientos de veces
en un \texttt{BNLJ}.

\newpage
\section{Referencias}
\begin{itemize}
	\item http://www.dba-oracle.com/phys\_52.htm
	\item \label{oracle_docs_memory}http://docs.oracle.com/cd/B10500\_01/server.920/a96533/memory.htm
	\item \label{objetos_pequenios}
\end{itemize}

\end{document}
