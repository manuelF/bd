\section{Implementaci\'on}

\subsection{Catalog Manager}

En primer lugar realizamos la implementaci\'on del manejador de c\'atalogo (\textit{Catalog Manager}). El propósito de este módulo es proveer datos para el resto de los módulos sobre las tablas de la base de datos. En particular se mantienen los siguientes datos sobre las tablas:

\begin{itemize}
	\item Identificador de tabla (que corresponde a un nombre de uso interno).
	\item Nombre de la tabla.
	\item El nombre de archivo que contendría los datos binarios de la tabla.
	\item El buffer al cual se deben asignar los datos de la tabla (valor opcional, se asigna un valor
	por \texttt{default} seg\'un la sem\'antica de buffers). 
\end{itemize}

La implementaci\'on del \textit{Catalog Manager} sigue la interfaz dada por la c\'atedra en \texttt{CatalogManager}. La implementaci\'on mantiene los cat\'alogos mediante el uso de archivos XML. A continuaci\'on presentamos un ejemplo de archivo:

\begin{Verbatim}[xleftmargin=-3em]
	<catalog>
		<table>
			<tableId>test.table</tableId>
			<tableName>A Test Table</tableName>
			<tablePath>testTable.table</tablePath>
			<tablePool>RECYCLE</tablePool>
		</table>
		<table>
			<tableId>test2.table</tableId>
			<tableName>Another Test Table</tableName>
			<tablePath>testTable2.table</tablePath>
		</table>
	</catalog>
\end{Verbatim}

Como se puede ver el formato es autoexplicativo: Se mantiene una secci\'on \textit{table} por cada tabla dentro de la cual se incluyen los datos pertinentes a esa tabla.

El único rol de la clase \texttt{CatalogManagerImpl} es levantar el archivo XML y parsear sus datos. Para ello decidimos utilizar la librer\'ia \texttt{JDOM 1.1}. Esta librer\'ia, si bien es de m\'as bajo nivel que XStream, no implica que tenemos \textit{coupling} de nombres entre el formato del XML y los nombres de las propiedades de los descriptores de tabla (lo cual es requerido por la librer\'ia antes mencionada por el uso de  \textit{reflection}). Utilizando esta librer\'ia de bajo nivel mantenemos separadas la representaci\'on en XML y la representaci\'on de los descriptores de tabla en c\'odigo en Java.

Una vez levantado el cat\'alogo desde un archivo XML, la interfaz que provee permite, dada una tabla, obtener el descriptor de tabla. Esto lo hace instanciando un objeto de clase \texttt{Catalog}, que contiene una lista de \texttt{TableDescriptors}, la representaci\'on de descriptores de tabla. Este contiene todos los datos mencionados anteriormente. En caso de no existir un descriptor correspondiente a la tabla, se devuelve un valor nulo (\texttt{null} en Java).

Junto a la implementaci\'on se incluyen test de unidad de este m\'odulo. Los mismos est\'an realizados mediante el framework JUnit. Consisten en levantar el cat\'alogo
de un archivo XML creado para el test y realizar luego una serie de verificaciones sobre los descriptores de tabla presentes, incluyendo casos borde como por ejemplo, un
descriptor de tabla que no tiene especificado en el XML a que pool debe ir (el comportamiento para este caso es enviarla al buffer \texttt{DEFAULT}) y que el valor correspondiente a un descriptor de tabla que no existe es nulo.

\subsection{Multiple Buffer Manager}

Siguiendo la l\'inea de la investigaci\'on realizada anteriormente, implementamos el manejo de buffers usado por el motor \texttt{Oracle 11g}.

Para lograr m\'as generalidad, decidimos implementar buffer pools multiples tal que puedan configurarse pools de nombres arbitrarios. Para ello, le permitimos al motor
inicializar pools a partir de un diccionario \texttt{Map<String,Integer>} que nos permita obtener los nombres y tama\~nos correspondientes para cada buffer deseado.

La configuraci\'on del buffer usado como default decidimos tambi\'en dejarla al instanciador permitiendo pasarla como par\'ametro (a pesar de que se podr\'ia utilizar un nombre como DEFAULT, lo cual corresponde a la implementaci\'on de Oracle). Tambi\'en permitimos definir la estrategia de reemplazo de p\'aginas para usar en los buffers. Por simplicidad se decidi\'o que todos los buffers utilicen esta estrategia pasada como par\'ametro.

La implementaci\'on es sencilla, puesto que se pudo reutilizar mucho c\'odigo ya provisto: cada buffer pool se modela mediante un \texttt{SingleBufferPool}. Se mantiene
un \texttt{Map<String,BufferPool>} que mapea $\text{nombre de pool} \rightarrow \text{instancia de pool}$. Estos mapeos se obtienen del cat\'alogo (que recibe el MultipleBufferPool de par\'ametro). La \'unica tarea del \texttt{MultipleBufferPool} por lo tanto es, dada una operaci\'on para una p\'agina, obtener el buffer correspondiente (que podría ser el DEFAULT si esta tabla no tiene asignado ningun pool) y realizar esa operaci\'on mediante \texttt{SingleBufferPool}.

Como se realizo anteriormente, se incluyen test de unidad para este m\'odulo realizados en JUnit. Para realizar los mismos se provee un \textit{mock} de la implementaci\'on de un cat\'alogo, que carga una lista est\'atica de descriptores.

