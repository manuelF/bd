\section{Conclusiones}

En este trabajo pr\'actico se investig\'o el manejo de buffers
como m\'odulo de un \textit{DBMS}, usando UBADB como plataforma de estudio. 

Se implement\'o la pol\'itica de \textit{Multiple Buffer Pools} bas\'andonos en la informaci\'on conocida del motor de bases de datos de Oracle, incluyendo adem\'as la implementaci\'on de un manejador b\'asico de cat\'alogo. 

Posteriormente a implementar y verificar la correctitud de estos m\'odulos (mediante tests de unidad) se procedi\'o a investigar y comparar esta estrategia de buffers contra la estrategia (ya implementada) de \textit{Single Buffer Pool} en una serie de casos de inter\'es. 

Se obtuvo que la performance del \textit{Multiple Buffer Pool} es similar o mejor a la del \textit{Single Buffer Pool} con una cantidad comparativa o incluso significativamente menor de recursos (en particular, se considero a este recurso como la cantidad de memoria principal necesaria). Concluimos entonces que la estrategia de \textit{Multiple Buffer Pools} es \'util para diversas situaciones, evitando con un gasto razonable de recursos la disminuci\'on de la \textit{performance} (en t\'erminos de accesos a disco por sobre accesos a memoria, que dada su diferencia de tiempo de acceso consideramos el \textit{hit-rate} como una buena medici\'on de \textit{performance}).
