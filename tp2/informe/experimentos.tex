\section{Experimentos}


Para plantear los experimentos que ibamos a correr, deberiamos simular los accesos a disco que serian ejecutados
en este motor de base de datos para resolver consultas.

Los experimentos buscaron replicar accesos comunes, donde hay tablas que son mas requeridas que otras y, por
consiguiente se deberian tratar de forma distinta a nivel de Buffer Manager. Como implementamos un mecanismo
similar al que utiliza Oracle con el Buffer Manager, vamos a simular accesos multiples y contar cuantas veces
accede a memoria principal a buscarlas, y cuantas veces deber\'a pedir p\'aginas f\'isicas del disco.

Para correr los experimentos se utiliz\'o el framework de generaci\'on de trazas incluido en el c\'odigo provisto
por la c\'atedra. Se generaron las trazas que pudieran simular accesos reales a tablas de forma directa (filescan),
indices clustered y unclustered, joins por BNLJ y mixtos (donde se simulan accesos concurrentes, que no respeten
un orden estricto).

Para poder investigar cual fue el hit rate de accesos al disco, se utilizaron las estadisticas que se pudieron
obtener de las clases Spy que hacian de proxy entre el driver de disco y la funci\'on que hacia el request
correspondiente. 

El fin de estos experimentos es poder visualizar las diferencias de performance que se pueden obtener si a la
hora de designar los buffers en memoria lo hacemos de una forma m\'as inteligente, en multiples Buffer Pools,
mejora el hitrate comparado con un \'unico Buffer Pool.
