\section{Funcionalidad}

\begin{itemize}
	\item Primera funcionalidad:
		\begin{itemize}
			\item Enunciado: Mediante SQL escribir una consulta para obtener el nombre, apellido y nombre
			identificatorio de aquellos pasajeros que han viajado a todos los paises cubertis por la l\'inea
			a\'erea en los \'ultimos 5 a\~nos.

			\item Resoluci\'on:

			\lstinputlisting{../tablas/ex1.sql}
		\end{itemize}
	\item Segunda funcionalidad:
		\begin{itemize}
			\item Enunciado: Controlar mediante alguna restricci\'on que un usuario no pueda realizar
			reservas que se superpongan en el tiempo. La \'unica excepci\'on que se permite consiste en
			que un usuario puede realizar a lo sumo dos reservas para la misma fecha de viaje entre los
			mismos aeropuertos de origen y destino, siempre que la fecha de partida no se encuentre dentro
			de los pr\'oximos 7 d\'ias.
			\item Resultado: Vamos a utilizar un \textit{trigger}:

			\lstinputlisting{../tablas/ex2.sql}
		\end{itemize}
	\item Tercera funcionalidad:
		\begin{itemize}
			\item Enunciado: Implementaci\'on de alguna restricci\'on adicional que surga del dise\~no.			
			\item Resultado: Decidimos implementar la siguiente restricci\'on:

			\begin{quotation}
				Para cada viaje, la cantidad de asientos reservados para ese viaje (considerando tambi\'en los vuelos
				que no reservaron directamente sino que es solamente una escala en su viaje) para una clase determinada 
				no supera la cantidad de asientos para esa misma clase en la aeronave.
			\end{quotation}

			Esta restricci\'on la implementamos mediante un \textit{trigger} en la inserci\'on de una reserva. 
			
			\lstinputlisting{../tablas/ex3.sql}
		\end{itemize}
	
	\item Cuarta funcionalidad:
		\begin{itemize}
			\item Enunciado: Obtener un reporte que como m\'inimo contenga: Todos los c\'odigos identificatorios de los
			aeropuertos, un per\'iodo de tiempo de la forma a\~no/mes, la cantidad de pasajeros que ascendieron y
			descendieron en ese aeropuerto durante el per\'iodo. El reporte debe estar ordenado por la cantidad total de personas
			que viajaron. Y debe ser ejecutado para un rango de fechas. Aclaraci\'on: dependiendo del rango de fechas, para
			un mismo aeropuerto pueden aparecer varios per\'iodos distintos.

		\item Resultado: Para este caso usamos tambi\'en un \textit{stored procedure}.
			Para producir este reporte, consideramos que un pasajero pis\'o un aeropuerto si tiene una reserva
			para un vuelo con escalas tal que tiene un vuelo que llega (o sale de, seg\'un corresponda) entre las fechas que se
			pasan como par\'ametro. Si un vuelo con escalas tiene varias escalas en ese aeropuerto, se considera una vez.

			\lstinputlisting{../tablas/ex4.sql}
		\end{itemize}

	\item Quinta funcionalidad:
		\begin{itemize}
			\item Enunciado: Para las reservas superpuestas que se permiten en el punto anterior (ver funcionalidad n\'umero
			2), se debe contar con una funcionalidad que cancele una de las reservas duplicadas (la m\'as econ\'omica) cuando
			la fecha de partida se encuentre dentro de los pr\'oximos 7 d\'ias.

		\item Resultado: Implementamos la funcionalidad como un \textit{stored procedure}. El mismo recibe el id de una reserva y
			elimina la reserva m\'as barata que coincide con ella que se encuentre dentro de los pr\'oximos 7 d\'ias de la
			misma y que tenga el mismo aeropuerto de salida y llegada, y la misma fecha de partida del primer vuelo y de
			llegada del \'ultimo vuelo. La raz\'on por la que se implement\'o esto as\'i es porque el caso de uso que vimos para
			esta funcionalidad es, dado que un usuario quiere registrar una reserva nueva, se le informa que tiene otra reserva
			que se superpone a esa y entonces indica que quiere cancelarla.

			\lstinputlisting{../tablas/ex5.sql}
		\end{itemize}

\end{itemize}

