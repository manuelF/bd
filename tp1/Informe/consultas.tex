\section{Funcionalidad}

\begin{itemize}
	\item Primera funcionalidad:
		\begin{itemize}
			\item Enunciado: Mediante SQL escribir una consulta para obtener el nombre, apellido y nombre
			identificatorio de aquellos pasajeros que han viajado a todos los paises cubertis por la l\'inea
			a\'erea en los \'ultimos 5 a\~nos.

			\item Resoluci\'on:

			\begin{lstlisting}
				SELECT nombre, apellido, idUsuario FROM usuarios
				WHERE idUsuario IN (
					SELECT idUsuario FROM usuarios WHERE NOT EXISTS (
						SELECT p.idPais FROM paises WHERE NOT EXISTS (
							SELECT idReserva FROM reservas r
							JOIN esReservaPara ep ON r.idReserva = ep.idReserva
							JOIN viajeConEscalas vce 
								ON vce.idViajeConEscalas = ep.idViajeConEscalas
							JOIN viaje v ON v.idViaje = vce.idViajeLlegada
							JOIN aeropuerto ap ON v.llega = ap.idAeropuerto
							JOIN ciudades c ON c.idCiudad = ap.idCiudad
								WHERE c.idPais = p.idPais
						)
					)
				)
			\end{lstlisting}
		\end{itemize}
	\item Segunda funcionalidad:
		\begin{itemize}
			\item Enunciado: Controlar mediante alguna restricci\'on que un usuario no pueda realizar
			reservas que se superpongan en el tiempo. La \'unica excepci\'on que se permite consiste en
			que un usuario puede realizar a lo sumo dos reservas para la misma fecha de viaje entre los
			mismos aeropuertos de origen y destino, siempre que la fecha de partida no se encuentre dentro
			de los pr\'oximos 7 d\'ias.
			\item Resultado: Vamos a utilizar un \textit{trigger}:

			\begin{lstlisting}
				CREATE TRIGGER RestringirReservasUsuario ON Reservas
				BEFORE INSERT
				FOR EACH ROW
				BEGIN
					IF( SELECT ( :NEW 
						RAISE ERROR ('Restriccion de reservas violada')
						ROLLBACK TRANSACTION
				END
			\end{lstlisting}
	
			\end{itemize}
\end{itemize}

