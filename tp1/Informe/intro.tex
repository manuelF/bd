\section{Introducci\'on}

En este trabajo pr\'actico se detalla el modelado realizado para
el \textit{backend}, en particular el subsistema de almacenamiento 
de datos mediante una base de datos relacional, de un sistema online
de reserva de pasajes de una aerol\'inea (problema referido en el enunciado 
bajo el nombre de \textit{booking}). Este problema consiste en mantener 
un sistema de reservas de pasajes de manera que el sistema en su totalidad 
(es decir m\'as alla del m\'odulo de base de datos implementado) permita a los
usuarios consultar y administrar reservas para pasajes de avi\'on.

En primer lugar se detalla el DER (Diagrama Entidad Relaci\'on) realizado
para este trabajo, junto con las limitaciones adicionales al modelo expresadas
en lenguaje coloquial. Posteriormente se detalla el Modelo Relacional derivado
del Diagrama Entidad Relaci\'on. Finalmente, se detalla la implementaci\'on
(mediate SQL) de 3 funcionalidades pedidas en el enunciado del trabajo pr\'actico.
Las mismas se detallan acordemente antes de mostrarse su implementaci\'on.
