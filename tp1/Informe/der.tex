\section{Diagrama Entidad Relaci\'on}

\subsection{Diagrama en si}

A continuaci\'on incluimos el Diagrama Entidad Relaci\'on (DER) correspondiente
al dise\~no realizado para este trabajo. El mismo se puede ver en la Figura~\ref{fig::der}
(dado el tama\~no del mismo y para comodidad de an\'alisis se incluye al final de este trabajo).

\subsection{Restricciones adicionales en lenguaje coloquial}

Las siguientes limitaciones adicionales del Diagrama Entidad Relaci\'on mostrado anteriormente,
y que no se pueden detallar utilizando la notaci\'on misma, se incluyen a continuaci\'on.

\begin{enumerate}
	\item Un usuario puede tener hasta 3 ciudades favoritas (restricci\'on tomada del enunciado).
	\item Los \'indices de escala de un vuelo con escalas son consecutivos. Adicionalmente, se cumple
	que los aeropuertos de llegada y de salida es encadenan (esta restricci\'on, si bien no es necesaria
	la tomamos por simplicidad). Por ``se encadenan'' se entiende que, dada la escala con \'indice $i$
	ocurre que
	
	\begin{itemize}
		\item $i = 0$ o el aeropuerto de llegada de la escala $i-1$ es el aeropuerto de salida de
		la escala $i$. Si $i = 0$ el aeropuerto de salida tiene que coincidir con el aeropuerto de llegada
		del vuelo que sale.
		\item $i = n-1$ con $n$ la cantidad de escalas del viaje con escalas considerado, o el aeropuerto
		de llegada de la escala $i$ es el aeropuerto de salida de la escala $i+1$. Si $i = n-1$ entonces
		el aeropuerto de llegada de esa escala debe concidir con el aeropuerto de salida del vuelo que termina.
	\end{itemize}

	En particular esto quiere decir, que un vuelo sin escalas tiene el mismo vuelo de salida y llegada.
	
	\item El precio de un vuelo esta determinado por la clase y el viaje. Se asume que los vuelos
	intermedios se hacen en la misma clase y esto no tiene injerencia directa en el precio (no se guardan precios intermedios).
	\item Para cada viaje dentro de los viajes de salida llegada y escalas de un viaje con escala, la cantidad de asientos 
	reservados para ese viaje para una clase determinada no supera la cantidad de asientos para esa misma clase en la aeronaves
	que realizan los viajes. Es decir no se puede sobrevender pasajes para ninguna clase.
    \item El usuario unicamente podr\'a tener cargada una tarjeta a la vez en el sistema.
\end{enumerate}

\subsection{Explicaci\'on de las decisiones tomadas}

A continuaci\'on detallamos el por qu\'e de algunas de las decisiones tomadas para el diseño del Diagrama Entidad Relaci\'on.

\begin{enumerate}
	\item Se consider\'o, como se explico anteriormente, que un viaje con escalas consiste de un viaje con varias escalas. El
	precio del viaje esta determinado por la relaci\'on entre el vuelo y la clase con la que se realiza el vuelo. Esto implica
	que se asume que los viajes intermedios (las escalas) se realizan en la misma clase que el viaje. Esto se hizo en base a que
	simplificaba el modelo y no encontramos un punto del enunciado que requiriera mantener las clases para viajes intermedios.
	Esto sin embargo produce que para vuelos sin escalas, implicitamente deba ocurrir que el vuelo de llegada y salida sean el
	mismo vuelo. El resultado fue elegido despu\'es de analizar los \textit{tradeoffs} de implementaci\'on.
	\item Se consider\'o preferible mantener la informaci\'on de los datos del viajero separado en cada reserva: La motivaci\'on
	de esto es que se ve por que el enunciado dice que una reserva puede ser realizada en nombre de otra persona (sin que esa
	persona sea un usuario registrado). Sin embargo, se le cobrar\'a a la persona registrada que realiz\'o la reserva (usando
	para ello los datos de cobro que tiene el usuario registrado). Si bien esto puede introducir redundancia cuando un usuario
	saca una reserva para si mismo, la otra opci\'on ser\'ia mantener valores especiales. Preferimos mantener una duplicaci\'on
	(se puede para ello usar un \textit{trigger} en los inserts) para mantener la simplicidad de los datos (las otras opciones
	consideradas implicaban mantener posibles nulls lo cual implicaba mantener una l\'ogica impl\'icita independiente de los
	datos guardados).
	\item Se decidi\'o que las multiples reservas de un usuario se modelan como reservas separadas. Se decidi\'o adicionalmente
	modelar que una reserva corresponde a solamente un vuelo (dentro de lo que se consideran los datos del backend del WIS).
	\item Algunos de los detalles del enunciado, a falta de una especificaci\'on m\'as fina de los datos, se decidi\'o mantenerlos
	como campos de texto con contenido indeterminado. Ejemplos incluyen: la composici\'on de la tripulaci\'on, los datos de llegada
	y salida de un aeropuerto, los datos de la persona a nombre de cual se hace una reserva, la tasa de un aeropuerto, el modelo
	de una aeronave, etc.
	\item El tel\'efono de Aeropuerto se guarda como una entidad d\'ebil (y no como argumento) porque consideramos 
	que un aeropuerto puede tener varios tel\'efonos (a diferencia de un usuario, de quien consideramos s\'olo uno).
    \item La unicidad de la tarjeta se desprende de simplificar el sistema de facturaci\'on, que no es la parte crucial de
    modelar. Se podria arreglar vinculando como atributo de la relacion en la reserva la tarjeta usada, pero
    agregaria complejidad convirtiendo la tarjeta en entidades fuertes que tienen que existir aunque el usuario no la use mas,
    como log historico.
\end{enumerate}
