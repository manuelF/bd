\section{Diagrama Entidad Relaci\'on}

\subsection{Diagrama en si}

A continuaci\'on incluimos el Diagrama Entidad Relaci\'on (DER) correspondiente
al dise\~no realizado para este trabajo. El mismo se puede ver en la Figura~\ref{fig::der}

\subsection{Restricciones adicionales en lenguaje coloquial}

Las siguientes limitaciones adicionales del Diagrama Entidad Relaci\'on mostrado anteriormente,
y que no se pueden detallar utilizando la notaci\'on misma, se incluyen a continuaci\'on.

\begin{itemize}
	\item Un usuario puede tener hasta 3 ciudades favoritas (restricci\'on tomada del enunciado).
	\item Los \'indices de escala de un vuelo con escalas son consecutivos. Adicionalmente, se cumple
	que los aeropuertos de llegada y de salida es encadenan (esta restricci\'on, si bien no es necesaria
	la tomamos por simplicidad). Por ``se encadenan'' se entiende que, dada la escala con \'indice $i$
	ocurre que
	
	\begin{itemize}
		\item $i = 0$ o el aeropuerto de llegada de la escala $i-1$ es el aeropuerto de salida de
		la escala $i$.
		\item $i = n-1$ con $n$ la cantidad de escalas del viaje con escalas considerado, o el aeropuerto
		de llegada de la escala $i$ es el aeropuerto de salida de la escala $i+1$.
	\end{itemize}
	
	\item El vuelo esta determinado no solo por el vuelo en si sino tambi\'en por la clase del asiento. No se
	hacen otras distinciones entre asientos dentro de una misma clase ni nada por el estilo.
	\item El precio de un vuelo esta determinado por la clase y el viaje, para cada uno de los viajes que corresponden
	a las escalas del vuelo.
	\item La cantidad de reservas que tienen un vuelo bajo una cierta clase dentro de su lista (considerando escalas)
	no puede superar la cantidad de asientos disponibles para esa clase por la aeronave del vuelo.
\end{itemize}

\subsection{Explicaci\'on de las decisiones tomadas}

A continuaci\'on detallamos el por qu\'e de algunas de las decisiones tomadas para el diseño del Diagrama Entidad Relaci\'on.

\begin{itemize}
	\item Se considero, como se explico anteriormente, que un viaje con escalas consiste de un viaje con varias escalas. El
	precio del viaje esta determinado por las relaci\'on entre los vuelos y clases de los asientos para cada uno los varios 
	vuelos del viaje, considerando adem\'as las tasas de llegada de los aeropuertos de llegada y salida.
	\item Se considero preferible mantener la informaci\'on de los datos del viajero separado en cada reserva: La motivaci\'on
	de esto es que se ve por que el enunciado dice que una reserva puede ser realizada en nombre de otra persona (sin que esa
	persona sea un usuario registrado). Sin embargo, se le cobrar\'a a la persona registrada que realiz\'o la reserva (usando
	para ello los datos de cobro que tiene el usuario registrado). Si bien esto puede introducir redundancia cuando un usuario
	saca una reserva para si mismo, la otra opci\'on ser\'ia mantener valores especiales. Preferimos mantener una duplicaci\'on
	(se puede para ello usar un \textit{trigger} en los inserts) para mantener la simplicidad de los datos (las otras opciones
	consideradas implicaban mantener posibles nulls lo cual implicaba mantener una l\'ogica impl\'icita independiente de los
	datos guardados).
	\item Se decidi\'o que las multiples reservas de un usuario se modelan como reservas separadas. Sin embargo, una reserva
	puede comprender varios vuelos.
	\item Algunos de los detalles del enunciado, a falta de una especificaci\'on m\'as fina de los datos, se decidi\'o mantenerlos
	como campos de texto con contenido indeterminado. Ejemplos incluyen: la composici\'on de la tripulaci\'on, los datos de llegada
	y salida de un aeropuerto, los datos de la persona a nombre de cual se hace una reserva, la tasa de un aeropuerto, el modelo
	de una aeronave, etc.
	\item El teléfono de Aeropuerto se guarda como una entidad débil (y no como argumento) porque consideramos que un aeropuerto puede tener varios teléfonos (a diferencia de un usuario, de quien consideramos sólo uno).
\end{itemize}	
