%BEGIN COPYPASTE EL INFORME DEL INFO
\documentclass[10pt, a4paper,english,spanish]{article}
\usepackage{subfig}

\parindent=20pt
\parskip=8pt
\usepackage[width=15.5cm, left=3cm, top=2.5cm, height= 24.5cm]{geometry}

%\usepackage{ccfonts,eulervm} 
\usepackage[T1]{fontenc}
\usepackage{epigraph}
\usepackage{amsmath}
\usepackage{amsfonts}
\usepackage{amssymb}
\usepackage{fancyhdr}
\usepackage[activeacute, spanish]{babel}
\usepackage{cancel}
\usepackage[utf8]{inputenc}
\usepackage{algorithm}
\usepackage{algpseudocode}
\usepackage{afterpage}
\usepackage{url}
\usepackage{fancyhdr}
\usepackage{listings}
\usepackage{ulem}
\usepackage{dashrule}
\usepackage{stmaryrd}
\floatname{algorithm}{Algoritmo}

\newtheorem{theorem}{Teorema}[section]
\newtheorem{lemma}[theorem]{Lema}
\newtheorem{proposition}[theorem]{Proposici\'on}
\newtheorem{corollary}[theorem]{Corolario}

\newcommand{\Var}{\textbf{var }}
\newcommand{\True}{\textbf{true }}
\newcommand{\False}{\textbf{false }}
\newcommand{\Break}{\textbf{break }}
\newcommand{\Continue}{\textbf{continue }}
\newcommand{\Param}{\textbf{param }}

\newenvironment{proof}[1][Demostraci\'on]{\begin{trivlist}
\item[\hskip \labelsep {\bfseries #1}]}{\end{trivlist}}
\newenvironment{definition}[1][Definici\'on]{\begin{trivlist}
\item[\hskip \labelsep {\bfseries #1}]}{\end{trivlist}}
\newenvironment{example}[1][Ejemplo]{\begin{trivlist}
\item[\hskip \labelsep {\bfseries #1}]}{\end{trivlist}}
\newenvironment{remark}[1][Observaci\'on]{\begin{trivlist}
\item[\hskip \labelsep {\bfseries #1}]}{\end{trivlist}}

\newcommand{\qed}{\nobreak \ifvmode \relax \else
      \ifdim\lastskip<1.5em \hskip-\lastskip
      \hskip1.5em plus0em minus0.5em \fi \nobreak
      \vrule height0.75em width0.5em depth0.25em\fi}


      \newcommand{\imp}{\shortrightarrow}
      \newcommand{\imps}{$\shortrightarrow$}
\parindent 0em
\algrenewcommand{\algorithmiccomment}[1]{//\textit{#1} }

\lstset{language=sql,numbers=left,tabsize=2,
	morekeywords={BEGIN,DECLARE,FOR,CREATE,PROCEDURE,RAISEERROR,EACH,ROW,BEFORE,AFTER,MINUS,IF},
	breaklines=true,breakatwhitespace=true}

\pagestyle{fancy}
\thispagestyle{fancy}
\addtolength{\headheight}{1pt}
\lhead{Bases de Datos}
\rhead{Apunte de normalizaci\'on}
\cfoot{\thepage}
\renewcommand{\footrulewidth}{0.4pt}
\newcommand{\hblacksquare}{\hfill \blacksquare}
%FIN COPYPASTE EL INFORME DEL INFO
\begin{document}

\section{Normalizaci\'on}

\subsection{Modelo Relacional}

$R$: Esquema de relaci\'on

$A_{1}, A_{2}, ..., A_n$ o $A, B, C, ..., etc$ atributos de $R$

$r$: Instancia de $R$

$t_1, t_2, ..., etc$ tuplas de r

\section{Dependencias Funcionales (DF)}
\subsection{Definicion de DF}

Decimos que vale $X \imp Y en R \iff t_1(X)=t_2(X) \imp t_1(Y)=t_2(Y)$ 

Decimos que ``$X$ determina funcionalmente a $Y$'', o que ``$Y$ es determinado funcionalmente por $X$.

$X$, conjunto de atributos de $R$, lado izquierdo.
$Y$, conjunto de atributos de $R$, lado derecho.
$X e Y$ no tienen que ser ecesariamente disjuntos.
$t_1 y t_2$ dos tuplas cualesquiera de $r$.

Si $r$ cumple con todas las dependencias funcionales entonces decimos que $r$ es LEGAL.

Las dependencias funcionales las establece el dise\~nador de la BD.

\subsubsection{Ejemplo}
Supongamos que queremos registrar para una facultad los datos personales de los alumnos, las materias en las que se inscribieron y los ex\'amenes que rindieron.

Para esto definimos el siguiente esquema de la relaci\'on:

\textit{FACULTAD (LU, NOMBRE, MATERIA, IFEC, EFEC, NOTA)}

Y el siguiente conjunto de dependencias funcionales $F$:

\begin{itemize}
    \item LU \imps NOMBRE (no puede haber dos alumnos con el mismo LU)
    \item LU, MATERIA \imps IFEC (se puede inscribir una sola vez en cada materia)
    \item LU, MATERIA, EFEC \imps NOTA (hay una sola nota por examen)
\end{itemize}

Nota: Al no estar LU, MATERIA \imps EFEC se puede rendir varias veces la misma materia.

\subsection{Inferencias de F}


Decimos que si $F$ INFIERE $f$ ($F \vDash f$) toda $r$ que satisface $F$ debe necesariamente satisfacer también $f$

\subsubsection{Ejemplo}
Del conjunto de dependencias funcionales del Ejemplo 1, podemos inferir:

$F \vDash$ LU, MATERIA \imps NOMBRE, IFEC

\subsection{Reglas de inferencia (Axiomas de Armstrong)}
\begin{enumerate}
    \item \textbf{Reflexividad}: $Y \subseteq X \Rightarrow X \imp Y$
    \item \textbf{Aumento}: $(\forall W) X \imp Y \Rightarrow XW \imp WY$
    \item \textbf{Transitividad}: $X \imp Y \land Y \imp Z \Rightarrow X \imp Z$
\end{enumerate}

De $1$ inferimos las triviales.

Se sigue que siempre se cumple que $X \imp X$.

Por inercia se tiende a pensar que $X \imp Y \Rightarrow Y \imp X$, pero esto, aunque a veces puede ser verdadero, en general es falso.

\end{document}
